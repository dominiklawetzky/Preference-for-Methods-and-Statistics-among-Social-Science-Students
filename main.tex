\documentclass[doc]{apa7}
\usepackage[utf8]{inputenc}
\usepackage{times}
\usepackage{amssymb}
\usepackage{graphicx}
\usepackage[ngerman]{babel}
\usepackage[nottoc]{tocbibind}
\usepackage{hyperref}
\usepackage{csquotes}
\usepackage{tabularx}

\usepackage[style=apa, backend=biber]{biblatex}
\addbibresource{references.bib}

\title{Statistik I: Praktikum}
\shorttitle{Dokumentation: Statistik-Praktikum}
\authorsnames[1, 2, 3, 4]{Antonina Cherniak, Dominik Lawetzky, Kiara Reibold, Alla Tyshchenko}
\authorsaffiliations{s6479392@stud.uni-frankfurt.de, d.lawetzky@stud.uni-frankfurt.de, s0125296@stud.uni-frankfurt.de, s6581245@stud.uni-frankfurt.de}



\begin{document}

\maketitle

\section{Vorüberlegungen}
Ziel ist die Erhebung von Daten zur Überprüfung einer im Vorhinein aufgestellten Hypothese. Unsere Kleingruppe möchte in diesem Zusammenhang untersuchen, inwieweit
\begin{itemize}
    \item ein \textbf{Unterschied} zwischen dem Geschlecht und der Präferenz für Forschungsmethoden und Statistik besteht,
    \item ein \textbf{Zusammenhang} zwischen der Persönlichkeitseigenschaft Gewissenhaftigkeit und der Präferenz für Forschungsmethoden und Statistik
\end{itemize}
besteht. Dafür soll eine Umfrage unter Studierenden der Sozialwissenschaften an der Goethe-Universität durchgeführt werden. In dieser Befragung werden das Geschlecht (nominalskaliert als männlich, weiblich, divers), das Alter als Kontrollvariable (verhältnisskaliert), die Präferenz für Forschungsmethoden und Statistik (dazu mehr bei Operationalisierung) sowie die Gewissenhaftigkeit als Persönlichkeitseigenschaft aus den \textit{Big Five} erfasst. 

Neben der Gewissenhaftigkeit könnten hier zahlreiche andere Persönlichkeitseigenschaften untersucht werden. Ein hohes Maß an Gewissenhaftigkeit korreliert mit außerordentlichem Engagement bei der Erfüllung von Arbeitsaufträgen und einer insgesamt gut strukturierten Arbeitsweise \autocite{jackson_what_2010}. Forschungsmethoden und Statistik erfordern in vielen Fällen eben dieses strukturierte Vorgehen. Dementsprechend ist zu überprüfen, ob Studierende, die eine starke Ausprägung von Gewissenhaftigkeit aufweisen, im Bereich Forschungsmethoden und Statistik auf weniger Schwierigkeiten stoßen und mehr Freude daran finden.

\section{Organisation und Zeitplan}
Um die Arbeit an dem Projekt möglichst transparent zu gestalten und bestmöglich kollaborativ (aus der Ferne) zu arbeiten, haben wir ein GitHub Repository\footnote{\url{https://github.com/dominiklawetzky/PsyBSc2_Pra_Dokumentation.git}} erstellt, in dem wir dokumentieren und die Rohdaten zur Verfügung stellen. In dem \textit{Read Me} des Repositories findet sich zudem ein detaillierter Zeitplan.

\section{Hypothesen}
Die Auswertung der Daten wird hypothesengeleitet stattfinden. Dabei stehen folgende zwei Hypothesen im Vordergrund:
 \begin{itemize}
     \item Es bestehen geschlechtsspezifische Unterschiede in der Präferenz für Methoden und Statistik.
     \item Bei Studierenden in den Sozialwissenschaften korrelieren die Gewissenhaftigkeit und die Präferenz für Methoden und Statistik.
 \end{itemize}


\section{Operationalisierung}
\paragraph{Gewissenhaftigkeit} Die Gewissenhaftigkeit (engl. \textit{conscientiousness}) gehört zu den sogenannten \textit{Big Five}. Dementsprechend gibt es zahlreiche vorgefertigte und hinreichend evaluierte Tests für diese Persönlichkeitseigenschaft. Da die häufig verwendeten Tests NEO-PI-R und NEO-FFI unter Urheberrecht stehen, ist von deren Verwendung abzusehen. Stattdessen können Items aus dem \textit{International Personality Item Pool}\footnote{\url{https://ipip.ori.org}} verwendet werden; insbesondere bieten sich hier die Tests H178, Q17, H153 und X19 an.

\paragraph{Präferenz für Forschungsmethoden und Statistik} Weit komplexer als die Operationalisierung der Persönlichkeitseigenschaft Gewissenhaftigkeit ist es, die Präferenz für den Bereich Forschungsmethoden und Statistik zu erfassen. Hierfür haben wir eigens Items erstellt, die auf einer Ordinalskala die Präferenz für den Bereich Forschungsmethoden und Statistik erfassen (\autoref{Itembatterie}). Abschließend wird das arithmetische Mittel der numerischen Relative zu den Itemantworten einer*eines Merkmalsträgers*in bestimmt, welches wiederum die Präferenz für Forschungsmethoden und Statistik angeben soll. Aus Platzgründen sind die Antwortmöglichkeiten in der Tabelle verkürzt; in dem Fragebogen werden wir die klassischen Antwortmöglichkeiten des Likert-Typs verwenden verwenden (trifft nicht zu, trifft eher nicht zu, ..., trifft zu).

Zu beachten ist, dass zwei Fragestellung invers formuliert ist und das numerische Relativ somit ebenfalls invertiert werden muss, um das empirische Relativ zu repräsentieren. In der Itembatterie haben wir dies durch die Invertierung der Antwortmöglichkeiten deutlich gemacht.

Uns ist bewusst, dass die Items in ihrer Aussagekraft unterschiedlich \textit{stark} sind. Dies werden wir allerdings nicht weiter berücksichtigen und alle Items gleich gewichten.

\section{Rekrutierung}
Rekrutiert werden sollen Studierende aus den Sozialwissenschaften an der Goethe-Universität Frankfurt. Dafür werden E-Mail- und WhatsApp-Verteiler sowie Facebook-Gruppen verwendet.

Die Teilnehmer*innen sollen möglichst gleichmäßig aufgeteilt männlich und weiblich sein. Wir sind uns des Problems bewusst, dass in vielen Sozialwissenschaften der Anteil an Frauen überproportional ist (dies gilt insbesondere für die Psychologie). Um Konfundierungen zu vermeiden, beschränken wir uns nicht nur auf Psychologie-Studierende.

Als Kontrollvariable sind neben dem Alter die Anzahl der Fachsemester, das Studienfach und die Anzahl der etwaig bereits in anderen Studienfächern absolvierten Fachsemester zu erfassen. 

\section{Auswertung}
Für der Auswertung des Datensatzes sollen gemäß der Hypothesen zwei Korrelationsanalysen durchgeführt werden. Darüber hinaus sollten diese Korrelationsanalysen separat bezogen auf die Studierenden aus den einzelnen Fächern der Sozialwissenschaften wiederholt werden, um auszuschließen, dass es signifikante Unterschiede zwischen den Studienfächern gibt, die die primäre Auswertung konfudieren könnten. Außerdem sollte deskriptiv-statistisch untersucht werden, inwieweit ein Zusammenhang zwischen den Geschlechterproportionen in einem Studienfach und den Auswertungsergebnissen besteht, um sicherzugehen, dass signifikante Unterschiede beim Geschlecht in der Präferenz für Methoden und Statistik \textit{nicht} durch die Studienfachzugehörigkeit verursacht sind. So wäre es denkbar, dass sich der hohe Frauenanteil in der Psychologie auf die Ergebnisse auswirkt, sofern dieser nicht durch den Einschluss anderer Studienfächer ausgeglichen wird.

\begin{table}[!htbp]
  \centering
    \begin{tabularx}{\textwidth}{Xl}
    \toprule
    Item & Bewertung im Bereich [1;5] \\
    \midrule
    Beim Lesen wissenschaftlicher Studien lege ich besonderen Wert darauf, die Methoden zu verstehen. & nein / eher nein / neutral / eher ja / ja \\
    \hline
    Wenn ich eine mathematische Formel nicht auf Anhieb verstehe, gebe ich schnell auf.  & ja / eher ja / neutral / eher nein / nein \\
    \hline
    Ich wende meine Kenntnisse aus der Statistik hin und wieder im Alltag an, um Probleme zu erklären oder Lösungen für Probleme zu finden. & nein / eher nein / neutral / eher ja / ja \\
    \hline
    Beim Verfassen von Hausarbeiten achte ich darauf, die fachspezifischen Schreibkonventionen einzuhalten. & nein / eher nein / neutral / eher ja / ja \\
    \hline
    Mir ist es wichtig, dass meine eigenen Ausarbeitungen methodisch sauber gestaltet sind. & nein / eher nein / neutral / eher ja / ja \\
    \hline
    Ich spiele mit dem Gedanken, nach dem Studium in den Bereich der quantitativen Forschung zu gehen. & nein / eher nein / neutral / eher ja / ja \\
    \hline
    Nach meinem Studium möchte ich in einem Beruf arbeiten, in dem ich Statistik nicht anwenden muss. & ja / eher ja / neutral / eher nein / nein \\
    \hline
    In meiner Freizeit beschäftige ich mich hin und wieder mit Statistikprogrammen oder Excel, um Alltagsprobleme zu lösen. & nein / eher nein / neutral / eher ja / ja \\
    \bottomrule
        \caption{Itembatterie zur Operationalisierung der Präferenz für Forschungsmethoden und Statistik}\label{Itembatterie}
    \end{tabularx}%
    
\end{table}%

\printbibliography

\end{document}


